\documentclass[12pt]{article}

\usepackage[T1]{fontenc}
\usepackage[polish]{babel}
\usepackage[utf8]{inputenc}
\usepackage{lmodern}
\selectlanguage{polish}

\usepackage{graphicx}
\usepackage{tabularx, booktabs}
\usepackage{fancyhdr} 
\usepackage{geometry}
\usepackage{hyperref}
\usepackage{listings}
\usepackage{float} 
\usepackage{subfigure}

\usepackage{a4wide}

\geometry{left=15mm,right=25mm,%
bindingoffset=10mm, top=20mm, bottom=20mm}
 


\renewcommand{\maketitle}{
\begin{titlepage}
\begin{table}[t]
\centering
\begin{tabular}[t]{lcr}
 \includegraphics[width=70pt,height=70pt]{PW} & POLITECHNIKA WARSZAWSKA & \includegraphics[width=70pt,height=70pt]{MiNI}\\
& WYDZIAŁ MATEMATYKI & \\
& I NAUK INFORMACYJNYCH &
\end{tabular}
\end{table}
\vspace*{3cm}
  \begin{center}
    \LARGE
    \textbf {Raport}\\
   \vspace*{2 cm}
\begin{table}[!htp]
\begin{tabular}{p{4cm}p{10cm}}
\textit{Przedmiot:} &\textbf {Warsztaty z technik uczenia maszyn} \\
\\
\textit{Projekt:} &\textbf {Santander Customer Transaction Prediction} \\
\\
\textit{Autorzy:} &\textbf {Mateusz~Bieńkowski \newline
	Katarzyna~Gołębiewska \newline
	Filip~Grajek \newline
	Sebastian~Sudra \newline
	Łukasz~Sznajder \newline
	Nikodem~Wiśniewski \newline 
 } \\
\\
\end{tabular}
\end{table}

\vspace{5 cm}
  \center{\small Warszawa, dnia \today}
\end{center}
\end{titlepage}
}

\begin{document}
\maketitle

\newpage

\section{Opis projektu}

Projekt realizowany w ramach konkursu \textit{Santander Customer Transaction Prediction}\cite{santanderkaggle}. Celem projektu jest opracowanie jak klasyfikatora, który na podstawie dostępnych danych o konsumencie wskaże, czy dokona on transakcji czy też nie. 

\section{Opis danych}

Do dyspozycji uczestników konkursu organizator przygotował dwa zanonimizowane zbiory danych: \textit{train.csv} oraz \textit{test.csv}. Oba zbiory danych posiadają po 200 tysięcy wierszy. Każdy wiersz posiada identyfikator \textit{ID\_code} oraz 200 anonimowych cech (\textit{var0, var1 ... var199}). Zbiór \textit{train.csv} posiada dodatkowo kolumnę \textit{target} będącą etykietą, przyjmuje ona wartości $0$ lub $1$. Dane z pliku \textit{train.csv} służą do trenowania i wstępnej oceny jakości modelu. Plik \textit{test.csv} służy do oceny modelu przez organizatora konkursu, klasy dla tego zbioru nie są publicznie znane. Zgłoszeniem do konkursu jest plik zestawiający \textit{ID\_code} każdego rekordu z pliku \textit{test.csv} z jego przewidywaną klasą.


\subsection{Przetwarzanie i analiza danych}

% liczba danych w każdej z klas

% brakujące dane

\subsubsection{EDA (Exploratory Data Analysis)}



\subsubsection{PCA (Principal Component Analysis)}



\section{Ocena modeli}

Sposób oceny modeli został narzucony przez organizatorów konkursu. Ze względu na nierównomierny podział klas, do oceny modeli wykorzystana jest miara AUC, którą można uzyskać poprzez policzenie pola pod krzywą ROC.


\section{Modele klasyczne}

\section{Modele niekonwencjonalne}

\section{Wyniki i podsumowanie}


\newpage

\section{Bibliografia}
\begin{thebibliography}{9}

\bibitem{santanderkaggle}
\url{https://www.kaggle.com/c/santander-customer-transaction-prediction}

\bibitem{exampple}
  John Example,
  \textit{Example title}.
  Exampler,
  1996.

\end{thebibliography}

\end{document}